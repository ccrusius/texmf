\input cwebmac
\let\ifpdf\relax
\input eplain
\secpagedepth=1
\newbox\cctanglebox
\N{0}{1}Baseline grid.
This package modifies \TeX{}'s page building routine so that all the
main baselines are on an equally spaced (by \hbox{\verbatim
\baselineskip|endverbatim}{}) grid.
This not only mimicks traditional typesetting, but also results
in a more pleasing document.
\unorderedlist
\li In multiple colum documents, where the problem is most visible,
it is not uncommon for baselines not
to align accross colums, resulting in an unbalanced layout.
\li In single page documents the problem is less obvious, but it
shows up both in facing pages -- where each page can be considered
to be one column in a double-column document --, and on the same
sheet of paper, if the baselines of the pages on that sheet do not align.
This last problem is more visible on more ``transparent'' paper, but
one can readily see it, for example, by looking at page~51 of the \TeX{}book
against light: although the baselines of page~51 and~52 start aligned,
towards the end of the page they are completely out of sync.
\endunorderedlist
\M{2}The solution for this problem seems to have eluded all trials so far,
and this is not surprising: one needs to change the page building process
to make it work. In particular, at every point when \TeX{} is considering
the addition of vertical material to the output box, we must filter it to
make sure said material would end up aligned to the baseline grid. This is
very hard to do (impossible?) in \TeX{} alone, but Lua\TeX{} has
enough hooks to make this possible.
\M{3}I will
describe the algorithm in more detail later, but the idea is as follows:
Firstly, we will keep track of the total height of what was already
added to the output box. Let us call that value~$b_h$.
Then, we process each element in the \TeX{} recent
vertical contribution list.
\unorderedlist
\li If the element is a glue with natural dimension~$g$, we remove all stretch
from it, and update~$b_h\leftarrow b_h+g$.
\li If the element is a horizontal list with height~$h_h$ and depth~$h_d$, we
insert enough glue~$g$ before it to make $(b_h + g + h_h)\bmod x=0$ (in other
words, to align its baseline with the grid), and update~$b_h\leftarrow b_h + g + h_d + h_h$.
\endunorderedlist
There are more things we need to do, but the basic idea should be clear:
we want to only feed \TeX's output box with things that are already aligned
to the baseline grid, and give \TeX{} no opportunity to change that by not
giving it any room to stretch. One could say we want to make the output
box look like an airplane's ``economy class'' section.
\N{2}{4}Using this package.
This package is one part of my \TeX{} macro set, and has to be used inside it.
This means you must be willing to use Eplain instead of \LaTeX, for example.
Making this package portable should not be too difficult if you wish to do so
-- if you do, please publish your work so others can use it.

If you still want to use this, clone my \hbox{\verbatim
texmf|endverbatim}{} tree from GitHub, at
\hbox{\verbatim
https://github.com/ccrusius/texmf|endverbatim}{}, make sure \TeX{} can find it,
and add the following to your document's preamble. The call
to \hbox{\verbatim
\ccgridsetup|endverbatim}{} is to make sure everything is
properly arranged: it is already called once when you include
the package, but if your preamble changes things after that
\hbox{\verbatim
ccgrid|endverbatim}{} may have to recompute some values.
\Y\B\4\setbox\cctanglebox\hbox{Enabling the baseline grid in your document}\X4:\box\cctanglebox\X${}\equiv{}$\6
\verbatim
\input| ccgrid|endverbatim\6
\verbatim
%| ...| rest| of| preamble| ...|endverbatim\6
\verbatim
\ccgridsetup|endverbatim\6
\fi
\N{0}{5}The package files.
\M{6}\B\4\setbox\cctanglebox\hbox{*{xxxccgrid.lua}}\X6:\box\cctanglebox\X${}\equiv{}$\6
\verbatim
|endverbatim\setbox\cctanglebox\hbox{Lua global variables}\X11:\box\cctanglebox\X\verbatim
|endverbatim\6
\verbatim
|endverbatim\setbox\cctanglebox\hbox{Lua functions}\X14:\box\cctanglebox\X\verbatim
|endverbatim\6
\verbatim
return| {| |endverbatim\setbox\cctanglebox\hbox{Lua module exports}\X15:\box\cctanglebox\X\verbatim
| }|endverbatim\6
\fi
\M{7}\B\4\setbox\cctanglebox\hbox{*{ccgrid.tex}}\X7:\box\cctanglebox\X${}\equiv{}$\6
\verbatim
|endverbatim\setbox\cctanglebox\hbox{\TeX{} package preamble}\X8:\box\cctanglebox\X\verbatim
|endverbatim\6
\verbatim
|endverbatim\setbox\cctanglebox\hbox{\TeX{} global variables}\X12:\box\cctanglebox\X\verbatim
|endverbatim\6
\verbatim
|endverbatim\setbox\cctanglebox\hbox{\TeX{} macros}\X13:\box\cctanglebox\X\verbatim
|endverbatim\6
\verbatim
|endverbatim\setbox\cctanglebox\hbox{Set up grid parameters}\X21:\box\cctanglebox\X\verbatim
|endverbatim\6
\verbatim
|endverbatim\setbox\cctanglebox\hbox{\TeX{} package postamble}\X9:\box\cctanglebox\X\verbatim
|endverbatim\6
\fi
\M{8}\B\4\setbox\cctanglebox\hbox{\TeX{} package preamble}\X8:\box\cctanglebox\X${}\equiv{}$\6
\verbatim
\input| ccbase|endverbatim\6
\verbatim
\pragmaonce{ccgrid}|endverbatim\6
\verbatim
\input| ccshowbox|endverbatim\6
\verbatim
\directlua{ccgrid| =| dofile(kpse.find_file("ccgrid.lua"))}|endverbatim\6
\verbatim
\makeatletter|endverbatim\6
\fi
\M{9}\B\4\setbox\cctanglebox\hbox{\TeX{} package postamble}\X9:\box\cctanglebox\X${}\equiv{}$\6
\verbatim
\makeatother|endverbatim\6
\verbatim
\endinput|endverbatim\6
\fi
\N{0}{10}Debugging.
Things did go wrong quite often with this, as baseline gridding is not something
\TeX{} was designed to do, so I had to set up a decent enough debugging
infrastructure. You will probably not be debugging this package, but the
code needs to be here anyway.
In order to control what debugging messages are
printed, you have to set \hbox{\verbatim
ccgrid|endverbatim}{}'s log level to a suitable value, as follows:

\halign{\strut\hfil{\tt #}\hfil&\quad#\hfil\cr
\omit\bf Level\hfil&\bf Information\hfil\cr
0 & Nothing.\cr
1 & Adds a baseline grid to every page.\cr
2 & Prints the contents of \hbox{\verbatim
\box255|endverbatim}{} every page.\cr
3 & Trace page building process.\cr}
\M{11}In Lua, the log level is stored in the global \hbox{\verbatim
loglevel|endverbatim}{} variable.
We will keep a copy of in in \TeX's \hbox{\verbatim
\ccgridloglevel|endverbatim}{} counter. To keep both in
sync, you should always change the log level by calling the \hbox{\verbatim
\setccgridloglevel|endverbatim}{}
\TeX{} macro.
\Y\B\4\setbox\cctanglebox\hbox{Lua global variables}\X11:\box\cctanglebox\X${}\equiv{}$\6
\verbatim
local| loglevel| =| 0|endverbatim\6
\fi
\M{12}\B\4\setbox\cctanglebox\hbox{\TeX{} global variables}\X12:\box\cctanglebox\X${}\equiv{}$\6
\verbatim
\newcount\ccgridloglevel\ccgridloglevel=0|endverbatim\6
\fi
\M{13}\B\4\setbox\cctanglebox\hbox{\TeX{} macros}\X13:\box\cctanglebox\X${}\equiv{}$\6
\verbatim
\def\setccgridloglevel#1{%|endverbatim\6
\verbatim
| | \directlua{ccgrid.setloglevel(#1)}%|endverbatim\6
\verbatim
| | \global\ccgridloglevel=#1}|endverbatim\6
\fi
\M{14}\B\4\setbox\cctanglebox\hbox{Lua functions}\X14:\box\cctanglebox\X${}\equiv{}$\6
\verbatim
local| function| setloglevel(x)| loglevel| =| x| end|endverbatim\6
\fi
\M{15}\B\4\setbox\cctanglebox\hbox{Lua module exports}\X15:\box\cctanglebox\X${}\equiv{}$\6
\verbatim
setloglevel| =| setloglevel,|endverbatim\6
\fi
\M{16}Most of the debugging messages are printed by the Lua module via a call
to the \hbox{\verbatim
typeout|endverbatim}{} function below:
\Y\B\4\setbox\cctanglebox\hbox{Lua functions}\X14:\box\cctanglebox\X${}\mathrel+\equiv{}$\6
\verbatim
local| function| typeout(lvl,str)|endverbatim\6
\verbatim
| | if| loglevel| >=| lvl| then| texio.write_nl(str)| end|endverbatim\6
\verbatim
end|endverbatim\6
\fi
\N{0}{17}Displaying a baseline grid.
To enable the display of a baseline grid at every
page, use the macro below. Doing this before a final build is a good idea, as
it will quickly tell you whether things are working or not. The macro works
by setting the appropriate log level, as described previously.
\Y\B\4\setbox\cctanglebox\hbox{\TeX{} macros}\X13:\box\cctanglebox\X${}\mathrel+\equiv{}$\6
\verbatim
\def\ccgriddraft{\setccgridloglevel{1}}|endverbatim\6
\fi
\M{18}To print a baseline grid at every page, we redefine the \hbox{\verbatim
\output|endverbatim}{}
routine.
\Y\B\4\setbox\cctanglebox\hbox{\TeX{} macros}\X13:\box\cctanglebox\X${}\mathrel+\equiv{}$\6
\verbatim
\newtoks\ccgrid@prevoutput|endverbatim\6
\verbatim
\ccgrid@prevoutput=\expandafter{\the\output}|endverbatim\6
\verbatim
\output={%|endverbatim\6
\verbatim
| | \ifnum\ccgridloglevel>0%|endverbatim\6
\verbatim
| | | | \setbox0=|endverbatim\setbox\cctanglebox\hbox{Baseline grid box}\X19:\box\cctanglebox\X\verbatim
|endverbatim\6
\verbatim
| | | | \setbox255=\vbox| to\vsize{\vtop| to0pt{\box0\vss}\hrule| height| 0pt\box255}|endverbatim\6
\verbatim
| | \fi|endverbatim\6
\verbatim
| | \the\ccgrid@prevoutput}|endverbatim\6
\fi
\M{19}\B\4\setbox\cctanglebox\hbox{Baseline grid box}\X19:\box\cctanglebox\X${}\equiv{}$\6
\verbatim
\vbox| to\vsize{|endverbatim\6
\verbatim
| | |endverbatim\setbox\cctanglebox\hbox{Baseline grid rule}\X20:\box\cctanglebox\X\verbatim
|endverbatim\6
\verbatim
| | \vskip\topskip|endverbatim\6
\verbatim
| | \cleaders\vbox| to\baselineskip{|endverbatim\6
\verbatim
| | | | |endverbatim\setbox\cctanglebox\hbox{Baseline grid rule}\X20:\box\cctanglebox\X\verbatim
|endverbatim\6
\verbatim
| | | | \vfil%|endverbatim\6
\verbatim
| | | | |endverbatim\setbox\cctanglebox\hbox{Baseline grid rule}\X20:\box\cctanglebox\X\verbatim
}|endverbatim\6
\verbatim
| | \vfill}|endverbatim\6
\fi
\M{20}\B\4\setbox\cctanglebox\hbox{Baseline grid rule}\X20:\box\cctanglebox\X${}\equiv{}$\6
\verbatim
\kern-0.2pt\hrule| height0.2pt| depth0.2pt| width\hsize\kern| -0.2pt|endverbatim\6
\fi
\N{0}{21}Parameter initialization.
A gridded \TeX{} run must have some parameters properly initialized.
This includes removing stretch from all known skips and sizing things
such as \hbox{\verbatim
\vsize|endverbatim}{} properly. All of this is done in the \hbox{\verbatim
\ccgridsetup|endverbatim}{}
macro, which is called automatically when \hbox{\verbatim
ccgrid.tex|endverbatim}{} is read.
\Y\B\4\setbox\cctanglebox\hbox{Set up grid parameters}\X21:\box\cctanglebox\X${}\equiv{}$\6
\verbatim
\def\ccgridsetup{|endverbatim\6
\verbatim
| | |endverbatim\setbox\cctanglebox\hbox{Remove glue from \hbox{\verbatim
\baselineskip|endverbatim}{} and set the Lua grid}\X28:\box\cctanglebox\X\verbatim
|endverbatim\6
\verbatim
| | |endverbatim\setbox\cctanglebox\hbox{Remove glue from other \TeX{} skips}\X29:\box\cctanglebox\X\verbatim
|endverbatim\6
\verbatim
| | |endverbatim\setbox\cctanglebox\hbox{Set \hbox{\verbatim
\vsize|endverbatim}{} to a multiple of \hbox{\verbatim
\baselineskip|endverbatim}{}}\X30:\box\cctanglebox\X\verbatim
|endverbatim\6
\verbatim
| | |endverbatim\setbox\cctanglebox\hbox{Set \hbox{\verbatim
\lineskip|endverbatim}{} and \hbox{\verbatim
\lineskiplimit|endverbatim}{}}\X31:\box\cctanglebox\X\verbatim
|endverbatim\6
\verbatim
| | \raggedbottom%| We| do| our| own| thing,| but| let's| tell| others| this| is| the| intent|endverbatim\6
\verbatim
}|endverbatim\6
\verbatim
\ccgridsetup|endverbatim\6
\fi
\M{22}The spacing for our grid should be essentially \hbox{\verbatim
\baselineskip|endverbatim}{}, but we need
to fix this value at the beginning of the document in a separate variable,
since \TeX{} may change it mid-course. In Lua, we keep the value in the
\hbox{\verbatim
baselineskip|endverbatim}{} variable, and the user can only change it by first
setting \hbox{\verbatim
\baselineskip|endverbatim}{} accordingly, and then calling the \TeX
\hbox{\verbatim
\ccgridseput|endverbatim}{} macro.
\Y\B\4\setbox\cctanglebox\hbox{Lua global variables}\X11:\box\cctanglebox\X${}\mathrel+\equiv{}$\6
\verbatim
local| baselineskip| =| 0|endverbatim\6
\fi
\M{23}\B\4\setbox\cctanglebox\hbox{Lua functions}\X14:\box\cctanglebox\X${}\mathrel+\equiv{}$\6
\verbatim
local| function| setgrid(x)| baselineskip=x| end|endverbatim\6
\fi
\M{24}\B\4\setbox\cctanglebox\hbox{Lua module exports}\X15:\box\cctanglebox\X${}\mathrel+\equiv{}$\6
\verbatim
setgrid| =| setgrid,|endverbatim\6
\fi
\M{25}Some basic math has to be performed when setting these parameters,
and that is done on the Lua side of things.
\Y\B\4\setbox\cctanglebox\hbox{Lua functions}\X14:\box\cctanglebox\X${}\mathrel+\equiv{}$\6
\verbatim
local| function| snapdown(x)|endverbatim\6
\verbatim
| | return| baselineskip*math.floor(x/baselineskip)|endverbatim\6
\verbatim
end|endverbatim\6
\verbatim
local| function| freeze(x)|endverbatim\6
\verbatim
| | return| (ccbase.spstr(ccbase.tosp(x))).."| plus| 0pt| minus| 0pt"|endverbatim\6
\verbatim
end|endverbatim\6
\fi
\M{26}\B\4\setbox\cctanglebox\hbox{Lua module exports}\X15:\box\cctanglebox\X${}\mathrel+\equiv{}$\6
\verbatim
snapdown| =| snapdown,|endverbatim\6
\verbatim
freeze| =| freeze,|endverbatim\6
\fi
\M{27}\B\4\setbox\cctanglebox\hbox{\TeX{} macros}\X13:\box\cctanglebox\X${}\mathrel+\equiv{}$\6
\verbatim
\def\ccgrid@freeze#1{\directlua{tex.print(ccgrid.freeze("#1"))}}|endverbatim\6
\fi
\M{28}\B\4\setbox\cctanglebox\hbox{Remove glue from \hbox{\verbatim
\baselineskip|endverbatim}{} and set the Lua grid}\X28:\box\cctanglebox\X${}\equiv{}$\6
\verbatim
\global\baselineskip=\ccgrid@freeze{\the\baselineskip}|endverbatim\6
\verbatim
\directlua{ccgrid.setgrid(ccbase.tosp("\the\baselineskip"))}|endverbatim\6
\verbatim
\typeout{ccgrid:| baselineskip=\the\baselineskip}|endverbatim\6
\fi
\M{29}\B\4\setbox\cctanglebox\hbox{Remove glue from other \TeX{} skips}\X29:\box\cctanglebox\X${}\equiv{}$\6
\verbatim
\global\topskip=\ccgrid@freeze{\the\topskip}|endverbatim\6
\verbatim
\global\parskip=\ccgrid@freeze{\the\parskip}|endverbatim\6
\verbatim
\global\abovedisplayskip=\ccgrid@freeze{\the\abovedisplayskip}|endverbatim\6
\verbatim
\global\belowdisplayskip=\ccgrid@freeze{\the\belowdisplayskip}|endverbatim\6
\verbatim
\global\abovedisplayshortskip=\ccgrid@freeze{\the\abovedisplayshortskip}|endverbatim\6
\verbatim
\global\belowdisplayshortskip=\ccgrid@freeze{\the\belowdisplayshortskip}|endverbatim\6
\fi
\M{30}\B\4\setbox\cctanglebox\hbox{Set \hbox{\verbatim
\vsize|endverbatim}{} to a multiple of \hbox{\verbatim
\baselineskip|endverbatim}{}}\X30:\box\cctanglebox\X${}\equiv{}$\6
\verbatim
\global\advance\vsize| by-\topskip|endverbatim\6
\verbatim
\global\vsize=\directlua{tex.print(%|endverbatim\6
\verbatim
| | ccbase.spstr(ccgrid.snapdown(ccbase.tosp("\the\vsize"))))}|endverbatim\6
\verbatim
\global\advance\vsize| by\topskip|endverbatim\6
\verbatim
\typeout{ccgrid:| vsize=\the\vsize}|endverbatim\6
\fi
\M{31}\B\4\setbox\cctanglebox\hbox{Set \hbox{\verbatim
\lineskip|endverbatim}{} and \hbox{\verbatim
\lineskiplimit|endverbatim}{}}\X31:\box\cctanglebox\X${}\equiv{}$\6
\verbatim
\global\lineskip=0pt|endverbatim\6
\verbatim
\global\lineskiplimit=-0.5\baselineskip|endverbatim\6
\fi

\bye