%% @texfile{
%%   author = "Cesar Crusius",
%%   date = "23 November 2014",
%%   filename = "ccbase.tex",
%%   docstring = "Base TeX macros",
%% }
%%%%%%%%%%%%%%%%%%%%%%%%%%%%%%%%%%%%%%%%%%%%%%%%%%%%%%%%%%%%%%%%%%%%%%%%%%%%%%%
%
% \PRAGMAONCE
%
% Usage: \pragmaonce<id>
%
% Expands to \endinput after the first time it is called. Used to make sure
% files are not included twice. See an example immediately after this
% definition.
%
% We define this before anything else because we want to use it immediately.
%
\def\pragmaonce#1{
  \csname pragmaonce@#1\endcsname%
  \global\expandafter\let\csname pragmaonce@#1\endcsname=\endinput
}
%%%%%%%%%%%%%%%%%%%%%%%%%%%%%%%%%%%%%%%%%%%%%%%%%%%%%%%%%%%%%%%%%%%%%%%%%%%%%%%
\pragmaonce{ccbase}
%%%%%%%%%%%%%%%%%%%%%%%%%%%%%%%%%%%%%%%%%%%%%%%%%%%%%%%%%%%%%%%%%%%%%%%%%%%%%%%
%
% \TYPEOUT
%
% From LaTeX.
%
\def\typeout{\immediate\write17}
%%%%%%%%%%%%%%%%%%%%%%%%%%%%%%%%%%%%%%%%%%%%%%%%%%%%%%%%%%%%%%%%%%%%%%%%%%%%%%%
\typeout{File `ccbase.tex' <2014/11/23>}
%%%%%%%%%%%%%%%%%%%%%%%%%%%%%%%%%%%%%%%%%%%%%%%%%%%%%%%%%%%%%%%%%%%%%%%%%%%%%%%
%
% CATCODES
%
% Directly from 'tugboat.cmn'
%
%  The following allow for easier changes of category.  These require that
%  the character be addressed as a control-sequence: e.g. \makeescape\/ will
%  make the / an escape character.
\def\makeescape#1{\catcode`#1=0 }
\def\makebgroup#1{\catcode`#1=1 }
\def\makeegroup#1{\catcode`#1=2 }
\def\makemath#1{\catcode`#1=3 }
\def\makealign#1{\catcode`#1=4 }
\def\makeeol#1{\catcode`#1=5 }
\def\makeparm#1{\catcode`#1=6 }
\def\makesup#1{\catcode`#1=7 }
\def\makesub#1{\catcode`#1=8 }
\def\makeignore#1{\catcode`#1=9 }
\def\makespace#1{\catcode`#1=10 }
\def\makeletter#1{\catcode`#1=11 }
\def\makeother#1{\catcode`#1=12 }
\def\makeactive#1{\catcode`#1=13 }
\def\makecomment#1{\catcode`#1=14 }

\def\makeatletter{\catcode`\@=11 }      % included for historical reasons
\chardef\other=12
\def\makeatother{\catcode`\@=\other}
%%%%%%%%%%%%%%%%%%%%%%%%%%%%%%%%%%%%%%%%%%%%%%%%%%%%%%%%%%%%%%%%%%%%%%%%%%%%%%%
%
% LUACODE
%
% Usage: \luacode <lua code> \endluacode
%
% Directly from LuaTeX's "Writing Lua in TeX," using the catcode routines
% defined above.
%
% To syntax highlight Lua code inside TeX, create a
% $VIMFILES/after/syntax/plaintex.vim file with the following contents:
%
%   unlet b:current_syntax
%   syn include @LUA syntax/lua.vim
%
%   syn region luatex matchgroup=contextIdentifier
%       \ start='\\luacode'
%       \ end='\\endluacode'
%       \ contains=@LUA
%
\def\luacode{%
  \bgroup
  \makeother\{
  \makeother\{
  \makeother\^^M
  \makeother\#
  \makeother\~
  \makeother\%
  \doluacode
}
\bgroup
\makeother\^^M %
\long\gdef\doluacode#1^^M#2\endluacode{\directlua{#2}\egroup}%
\egroup
%%%%%%%%%%%%%%%%%%%%%%%%%%%%%%%%%%%%%%%%%%%%%%%%%%%%%%%%%%%%%%%%%%%%%%%%%%%%%%%
\endinput
